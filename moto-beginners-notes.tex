\documentclass[12pt,a4paper]{article}

\usepackage[utf8x]{inputenc}
\usepackage[T2A]{fontenc}
\usepackage[russian,english]{babel}

\usepackage{geometry}
\geometry{left=3cm}
\geometry{right=1.5cm}
\geometry{top=1.5cm}
\geometry{bottom=1.5cm}

\usepackage[hidelinks]{hyperref}

\title{Заметки начинающего мотоциклиста}
\author{Алексей Морараш}
\date{Версия: 10 июня 2014 года}

\begin{document}

\maketitle

\renewcommand{\contentsname}{Содержание}

\tableofcontents

\clearpage

\begin{abstract}
В статье я излагаю результаты своих личных наблюдений и умозаключений
в процессе обучения нелёгкому и крайне увлекательному делу езды на
мотоцикле. В своё время у меня было мало возможностей спросить совета,
потому пришлось по многим граблям походить самостоятельно. Надеюсь,
после прочтения граблей на Вашем пути станет меньше.
\end{abstract}

\section{Отказ от ответственности}

Всё приведенное - это субъективное мнение и никоим образом не
претендует на самое адекватное освещение всех проблем и вопросов.

Статья не является руководством к действию или инструкцией,
и автор не несёт ответственности за причинённый ущерб либо
недополученную выгоду из-за использования данной статьи.

\section{Вступление}

Каждый сам решает, чем для него является мотоцикл. Трудно ответить
на вопрос ``Зачем оно надо?''. Единственно правильных ответов
здесь быть не может. Для кого-то --- покататься по треку, для кого-то ---
помесить на бездорожье, для кого-то --- попутешествовать. А для иных и
вовсе --- ``красиво отойти''.

Для начала развенчаю один из устойчивых мифов о мото. В фильмах
мотоцикл --- это круто, комфортно, с ветерком в лицо и полуголая
блондинка в качестве пассажира. На самом деле, мотоцикл --- это,
прежде всего, опасность, пыль, грязь, масло, бензин, окоченевшие
от дикого холода конечности или насквозь мокрая от пота одежда
(в зависимости от погоды). Иными словами, людям, ищущим комфорта,
не место в рядах мотоциклистов. Мотоцикл --- это дело настоящих
извращенцев :)

Подозреваю, что эта картина несколько смягчается в случае, если
решено использовать мотоцикл только по праздникам и в хорошую погоду
с целью прокатиться по району и обслуживание его целиком и полностью
возлагается на специально обученных людей за отдельную плату.

Я, к примеру, использую мотоцикл каждый день вне зависимости от
погодных условий, для езды на работу и обратно а также для деловых и
прогулочных поездок по стране.

В этой статье вопросы освещены в том порядке, в котором, на мой
взгляд, они должны решаться и в жизни:

\begin{enumerate}
\item обучение езде, получение категории;
\item покупка экипировки;
\item покупка мотоцикла.
\end{enumerate}

\section{Обучение, получение водительских прав}

Как показывает практика, обычная автошкола многому не научит.
Мне на практических занятиях по вождению мотоцикла в автошколе лишь
объяснили расположение узлов управления и показали, как включать
первую и трогаться с места. В результате я получил очень много
острых ощущений, когда впервые выехал в город (при перегоне мотика
к дому), а само обучение выполнял на улицах своего спального района
ночью, когда никто не мешает.

Так что помимо отдачи дани уважения автошколе и сдачи экзаменов на
получение категории, есть смысл записаться в тематическую мото-школу.

Согласно нашему законодательству, для управления двухколёсными
транспортными средствами с рабочим объемом двигателя до 50 куб. см.
или до 4кВатт необходима категория А1. Категория А необходима для
управления двухколёсными транспортными средствами с рабочим объемом
двигателя более 50 куб. см. или более 4кВатт и массой не более 400 кг
(т.е. снаряжённая Honda GoldWing по нашему законодательству мотоциклом
уже не является). Для управления мотоциклами весом более 400 кг,
мотоциклами с коляской, трициклами, квадроциклами необходимо открывать
категорию B1.

\section{Покупка экипировки}

Некоторые катают без шлема, в футболке и шлёпанцах, но мы же взрослые
люди и должны понимать, что это --- апогей идиотизма. Есть такая шутка:
``если ты считаешь, что шлем тебе не нужен, то шлем тебе не нужен''.
В англоязычной среде таких пилотов называют ``сквид'' (англ. squid ---
кальмар). Именно как кальмар выглядит после падения такое тело:
порванное в клочья об асфальт, с вывернутыми конечностями и т.п.
Экипировку надо купить \emph{до} покупки мотоцикла и жёстко настроить себя на
то, что без неё на мотоцикл садиться нельзя. Конечно, не стоит забывать
о том, что ни один экип, сколь бы дорогим он ни был, не спасёт тебя от
плачевной участи, если на дороге ты не дружишь с головой.

Ещё одна важная вещь: падают \emph{все}. Вне зависимости от условий и
стажа. Твоя задача --- \emph{всегда} быть готовым к падению (не морально,
а в плане экипа). Если ты защищён и ведешь себя на дороге адекватно,
то после падения вскочишь, поднимешь байк и поедешь дальше. Ну а все
те, кто считал себя крутыми гонщиками, в большинстве своём уже так не
считают. Товарищ был знаком с шайкой сорвиголов на спортах в Одессе.
Их было 20 с копейками. Через 5 лет их осталось двое.

Экипировка бывает разной в зависимости от области применения. На
кольцевых гонках, к примеру, используются комбезы, максимально
защищающие от истирания об асфальт. Для мотокросса используется
текстиль (ибо нет асфальта), но с защитой от переломов конечностей,
ключиц. Для дорог общего пользования, как правило, используют
компромиссы, обеспечивающие и безопасность при соприкосновении с
асфальтом, и удобство ношения.

Общие правила выбора экипа: покупай самое дорогое, что можешь себе
позволить (мне, к примеру, моя первая экипировка обошлась в 2000\$).
Предпочтение отдавай известным брендам. Чем ярче экип, тем лучше тебя
будет видно на дороге --- это положительно сказывается на пассивной
безопасности. Быть чёрным рыцарем, конечно, круто, но быть пёстрым
попугаем --- безопаснее. Наличие светоотражающих элементов
приветствуется.

В качестве отступления, импровизированные краш-тесты шлемов:

\begin{itemize}
\item какого-то безымянного китайца:
\url{http://www.youtube.com/watch?v=FkSKG3cuoJ8};
\item японского Arai:
\url{http://www.youtube.com/watch?v=xJxmGvh_bPE}.
\end{itemize}

Важное замечание по температурным режимам. Есть такая шутка: на
мотоцикле никогда небывает ``нормально'', а бывает либо жарко, либо
холодно. На самом деле степень комфорта зависит от потока набегающего
на пилота воздуха. Если стоять в пробке на солнцепёке, вспотеешь и в
футболке, не говоря уже о серъёзном экипе, а если ехать со скоростью
хотя бы 70 км/ч, то становится уже весьма комфортно (при условии, что
аэродинамика мотоцикла не обводит поток воздуха мимо тебя и в куртке
имеются вентиляционные отверстия).

В холодное время года (при температуре, близкой к нулю), в типичном
экипе ехать можно только в городском режиме старт-стоп. 15 минут
сколько-нибудь приличной скорости, и ты уже синий.

Важное замечание, касающееся атмосферных осадков. Не надейся, что ты
всегда будешь кататься посуху. Всегда стоит быть готовым к тому, что
ты попадёшь в дикий ливень, особенно в дальняках. Однажды, катаясь
по Киеву, я три раза въезжал в сильный ливень и три раза из него выезжал
на сухой асфальт, и всё это за час-полтора езды, при том в разных
районах города! Основная опасность от промокания одежды --- это
последующее за этим замерзание на ветру. При этом не важно, в какое
время года. Однажды я так попал прямо посреди лета. Когда у тебя
замерзают кисти рук и ноги, ты уже не в состоянии эффективно управлять
мотиком --- окоченевшими культями ты уже не сможешь живо жать рычаги и
переключать передачи. Внимание рассеивается на ``блеать, как же
холодно!'', и уже не до слежения за дорожной обстановкой.

Итак, надо быть готовым защитить себя от промокания. По личному опыту
скажу: это не достигается покупкой водонепроницаемой куртки, штанов и
перчаток. Куртки и штаны с этикеткой ``waterproof'', как правило, стоят
на 100\$ дороже, но водонепроницаемыми они будут от силы месяц-два, а
при сильном дожде рано или поздно промокнут до нитки --- я уже это
проходил. Реально защищает от промокания т.н. дождевой комплект. Цена
вопроса --- 100 грн в магазине для рыбалки. В него входят куртка на
молнии и штаны. Брать стоит максимальный размер, чтобы в дороге можно
было надеть прямо сверху, не снимая рюкзака и ботинок. В свёрнутом
состоянии занимает места, как кило картошки. Сверху на перчатки для
защиты от дождя надо купить т.н. дождевики --- они тонкие,
водонепроницаемые и не усложняют управление мотоциклом. Стоить могут
как очень дёшево, так и не очень (40\$). Дешёвые обычно раньше
утрачивают устойчивость против влаги. В таком снаряжении уже не страшен
никакой дождь --- мне доводилось ездить в лютейшие ливни.

Если нет уверенности в водонепроницаемости мотоботов, для защиты ног
можно приобрести специальные бахилы. Некоторые умудряются простые
пакеты из супермаркета в качестве бахил использовать.

\subsection{Шлем}

То, что носят на своих головах мотоциклисты, бывает разных видов:

\begin{itemize}
\item бандана. Говорят, правильно освящённая в храме бандана способна
остановить и пулю, но мы-то с вами знаем...;
\item обычная каска. Область применения --- гоняние понтов. Практическая
ценность стремится к нулю;
\item шлем открытого типа. Инспектора ГАИ удовлетворит, но с точки
зрения защиты головы далеко не лучшее решение;
\item шлем закрытого типа (интегральный). Обеспечивает наилучшую защиту;
\item шлем для мотокросса. Отличительные черты: козырёк, нет визора
(требует ношения специальных очков), выдающаяся вперёд защита челюсти.
\end{itemize}

Бывают ещё интегралы, но с откидывающейся передней частью. Сделано для
тех, кто ленится снимать/одевать шлем каждый раз, когда хочется выпить
кофе или покурить.

Из известных мне брендов: Shoei, Arai (Япония), Schubert (немцы).
Качественный шлем стоит от 600\$ и выше. В нормальном шлеме не шумно и
он обеспечивает адекватную вентиляцию (регулируемую).

На стоимость шлема влияет также и наличие аэрографии. Одна и та же
модель шлема может отличаться в зависимости от наличия рисунка на 50\$
и больше. Одноцветный шлем стоит дешевле разрисованного.

Шлем нельзя покупать с доставкой. Только в салоне! Его нужно подбирать
строго индивидуально. Прямо там надеваешь его и минут 15-20 в нём
ходишь. Ничего не должно чересчур давить, но он должен сидеть плотно.
Если шлем снимается одной рукой, он слишком велик.

Мерять шлем надо в подшлемнике, так как это тоже крайне необходимый
элемент экипа. Нормальный подшлемник станет примерно в 50\$, на те,
которые по цене носков даже не смотри. Подшлемник должен эффективно
защищать шею от продувания. К тому же намного проще периодически
стирать подшлемник, чем внутреннюю часть шлема.

Дополнительно рекомендую почитать статьи в сети по выбору и примерке
шлема, потому как покупка дорогая и надолго.

\subsection{Перчатки}

Перчатки должны быть тематическими --- с защитой. Связанные бабушкой
варежки не предлагать. Главное --- удобство. При покупке в салоне одень,
сядь на мотоцикл и попробуй поработать с рычагами управления и
тумблерами. Косяков быть не должно. Перчатки под себя я усиленно искал
недели две, пока нашёл такие, которые действительно сидели, как влитые
и не усложняли управление.

Взять стоит две пары: одни лёгкие и проветриваемые для тёплой погоды и
одни плотные, утеплённые --- для весны-осени (не забываем про
температурные режимы во время езды). Иногда приходится возить обе пары
прямо с собой. Лёгкие можно найти от 100\$, утеплённые чуть дороже и
стоят от 150\$.

\subsection{Мотоботы}

Армейские ботинки не подходят, проверено. Ни от холода, ни от воды не
спасут. Желательно брать боты повыше и с удобными долговечными
застёжками. Они должны быть жёсткими, чтобы защитить голеностоп в
случае чего. Мотоботы --- для езды, а не для ходьбы. Стоимость --- от 200\$.

\subsection{Куртка}

По выбору кожанок не подскажу, надо смотреть соответствующие статьи
в сети.

Если выбирать текстиль, то надо обратить внимание на:

\begin{itemize}
\item защита локтей, плеч, спины должна быть;
\item не должна быть слишком широкой, чтоб не хлопала на ветру.
В идеале должна быть возможность регулировать полноту запястий,
рукавов, торса;
\item крайне приветствуется наличие съёмной тёплой подстёжки;
\item ни рукава, ни низ не должны быть короткими. На моей, к примеру,
сзади низ удлиннён, чтоб спину закрывать в сидячем положении.
Застёгивающийся воротник тоже нужен --- шея наше всё. Простудишь --- не
сможешь головой вертеть;
\item должно быть достаточное количество регулируемых вентиляционных
карманов (у меня два на груди, по одному на рукавах и по два на каждом
боку);
\item карманов чем больше, тем лучше. Карманы должны уметь плотно
застёгиваться. Наличие водонепроницаемого кармана для документов
приветствуется.
\end{itemize}

Хорошая куртка обойдётся от 400\$.

\subsection{Штаны}

Я купил себе одни плотные (в комплекте с курткой) и одни мотоджинсы с
кевларовой подкладкой и наколенниками. Вывод: плотные штаны купил зря ---
они слишком тяжёлые, а летом в них вообще невозможно ездить.
Мотоджинсов вполне хватает на весь год. Мотоджинсы Sartso стали мне в
250\$.

Отдельно можно взять шарнирные наколенники (для надевания поверх
штанов) --- от 100\$.

\subsection{Где это всё купить?}

Самое удобное место --- МотоДом в Киеве, на Протасовом Яру, 13.
Преимущества:

\begin{itemize}
\item большой выбор. Можно за один раз придти и экипироваться с головы
до ног;
\item работники оказывают толковую помощь в подборе.
\end{itemize}

Недостатки:

\begin{itemize}
\item порой дороговато;
\item принципиально принимают только наличку, а единственный банкомат
в округе выдаёт только по 1500 грн за раз.
\end{itemize}

Есть ещё Internet-магазин MotoStyle где-то в центре, но выбор невелик ---
работают, в основном, под заказ. Но главное --- замечены в скотском
отношении к покупателю. Был свидетелем. Да и я неделю им названивал,
упрашивая таки привезти мне моторное масло. Лично я к ним больше ни
ногой.

Кроме того, есть ещё людишки, занимающиеся продажей экипировки у себя,
так сказать, прямо в гараже. Искать их надо в форумах на moto.kiev.ua.

\section{Покупка мотоцикла}

Каждый мотоцикл по-своему уникален и обладает кучей своих особенностей
в зависимости от своего предназначения. Многоцелевых мотоциклов не
бывает. Поэтому исключительно важно определиться, для каких именно
поездок он нужен.

Классификация мотоциклов порой является достаточно противоречивой.
Ни одна энциклопедия чёткой классификации, которая бы всех устроила,
не даёт, а я и подавно не претендую на качество. Здесь разбиваю
мотики на классы так, как это вижу сам:

\begin{itemize}
\item классика. Минимум пластикового обвеса (если его вообще нет и все
узлы торчат наружу, можно услышать термин ``нейкед'' --- от англ. ``naked'',
``голый''), почти прямая посадка. Примеры: Suzuki Bandit, Jawa 350;
\item крузер. Низкая по седлу прямая посадка, ``ноги вперёд''. Примеры:
Yamaha DragStar, Honda Shadow;
\item чоппер. Тот же крузер, но максимально упрощённый (в американском
стиле) --- может быть вообще без приборов и обвеса, маленький каплевидный
бак, длинная вилка и высокий размашистый руль --- это всё признаки
типичного чоппера.
\item спорт. Посадка пилота ``в позе эмбриона'', весь мотик затянут
пластиковыми обтекателями для уменьшения сопротивления воздуха при
рассекании на околозвуковых скоростях. Примеры: Honda Fireblade,
Suzuki GSX;
\item эндуро. Малый общий вес, радикально высокая подвеска и дорожный
просвет, посадка пилота ``стоя'', места для пассажира вообще как
правило не предусмотрено. Идеальный вариант для замеса говн в самых
непроходимых чащах. Бак маленький, запас хода до 100-150 км. Примеры:
Honda CRF, Suzuki RMX;
\item туристический эндуро (тур-эндуро). Говоря языком автомобилистов ---
``паркетник''. Обладает достаточно высокой подвеской, но тяжелее эндуро
и в большей степени предназначен для езды по дорогам. Примеры: Honda
Africa Twin, Honda Transalp, Yamaha Super Tenere, BMW GS;
\item туристический. Настоящий крейсер среди мотоциклов. Тяжёлый (порой
более 400 кг), мощный, но обеспечивающий наибольший комфорт для
пилота и пассажира. Может быть оборудован не только стерео-системой,
но и кондиционером (sic!), телевизором, радио и т.п. Стоит
соответственно. Примеры: Honda GoldWing.
\end{itemize}

В качестве примера, мой Honda CBR125R: идеален для города. Низкий
расход, малые габариты, лёгкий в управлении. Но для дальняков он мало
пригоден: производителем не предусмотрено навешивание кофров, а в
рюкзаке за спиной много не увезёшь --- устают плечи. Малый ход подвески
и относительно малый радиус дисков не способствует приятной езде по
нашим т.н. дорогам. Из-за отсутствия обтекателя тебя ``сдувает'', а
достаточно сильный встречный ветер из-за малой мощности двигателя
заставит тебя двигаться с позорными 80-90 км/ч. Сам мотоцикл чисто
дорожный, и вне дорог я бы на нём ездить не рискнул.
Из-за той же малой мощности двигателя приходится постоянно ехать на
пределе его возможностей, что, очевидно, не лучшим образом
сказывается на его износе. А если уж впереди окажется автомобиль,
едущий со скоростью 90-100 км/ч, то безопасно обогнать его порой
оказывается непосильной задачей --- запаса скорости нет.

Второй пример: Honda Transalp 650, который я взял для загородных поездок.
Его хвалят за надёжность, у него большой ход подвески, переднее колесо
радиуса 21 дюйм, наличие защиты двигателя, высокого стекла-обтекателя,
центральной подножки, креплений для центрального и боковых кофров делают
его весьма привлекательным для туристических поездок. Заявленной
производителем максимальной скорости в 170 км/ч с головой хватает для
умеренной езды и тех же обгонов.

Доводилось мне ездить и на крузерах Keeway SuperLight 150,
Hyosung Aquilla 650. Посадка ``вразвалочку'' располагает к спокойной езде.
Ощущения сильно отличаются от ощущений от езды на спорт-байках.

\subsection{Выбор первого мота}

Важно понимать, что опыт желательно получать постепенно. Для начала я
бы порекомендовал взять нечто такое, чтобы оно тебя не убило, и речь
даже не в гасании на запредельных скоростях. Мой Honda CBR125 --- один из
отличных вариантов для обучения (собственно, мой первый). Он лёгкий
(130 кг), и его достаточно легко удержать, когда он начинает
заваливаться (хотя один раз я его таки не удержал). Он лёгок в
управлении, как велосипед (образно выражаясь). Если же сразу взять
200-300 килограммовую литровую дуру, то одно неловкое движение, и она
либо валяется (удачный исход), либо ты начинаешь лечиться (тоже
неплохо), либо лечиться тебе уже не надо.
На моей хонде, к примеру, если ты бросишь сцепление, она заглохнет.
Мощный же байк просто из под тебя вылетит (или упадёт на тебя сверху).

Товарищ рассказывал, как взял тяжёлый мотик прокатиться, и совершая
поворот на перекрёстке, решил притормозить, пропуская пешехода.
Поскольку тормозить он начал в повороте и ещё до того, как выровнял
байк он его не удержал и упал. В общем, надо понимать, что серъёзная
техника ошибок новичка не прощает.

Подводя итог: в качестве первого я бы взял нечто с рабочим объёмом от
125 до 400 (правда, с объёмом 400 сейчас нового практически ничего не
выпускается) и весом в пределах 150 кг.

\subsection{Новый или б/у}

Тут всё очевидно. Некуда девать деньги --- бери в салоне, иначе изучай
тему оценки тех-состояния (а ещё лучше --- найди уже бывалого) и покупай
с рук. Для качественного ухоженного мота пробег в 50 тыс --- это ни о чём.

\subsubsection{Покупка в салоне}

Здесь всё просто. Если это японец/немец/итальянец, то напрягаться
нет причин. Обкатывай, как того требует инструкция.

\subsubsection{Покупка в салоне: китайский мотоцикл}

Если покупаешь какого-то китайца, всё намного интереснее. Дело в
том, что эти мотоциклы, как правило, не проходят вообще никакой
предпродажной подготовки. Их привозят в страну по частям, чтоб
сэкономить на пошлинах, и уже в стране, в каком-нибудь одесском
порту их собирают пьяные программисты. На дорогу на таком выезжать
опасно --- прямо по пути может что-нибудь открутиться или отвалиться.
Такая техника нуждается в предварительной настройке, проверке резьбовых
соединений, смазке, калибровке и т.п.

Но на этом не заканчиваются беды владельца China-байка. Этим
мотоциклам присущи несметные количества мелких и не очень косяков,
которые лечатся порой только заменой. Ниже привожу список того,
что мне уже довелось ``увидеть руками и пощупать глазами''.

\begin{itemize}
\item элементы управления либо хрен пойми как работают, либо размещены
не там, где человеческая рука их ожидает. Ручка газа, крутящаяся в обе
стороны, либо не возвращающаяся в исходное положение --- в порядке
вещей. Аварийка, которую невозможно включить, пока включен указатель
поворота. Указатели поворота, которые могут включаться независимо друг
от друга и многое-многое другое;
\item приборы могут показывать погоду на Марсе. Как показывает
практика, замена на ``вот эти вот хорошие'' помогает не всегда;
\item стоковая резина вообще не заслуживает упоминания. Выбросить
и заменить на нечто приличное, пока ещё живой;
\item стоковая цепь вполне может оказаться простой железкой без сальников,
соответственно ресурс у неё ≈ 5 тысяч пробега. В качестве примера:
первую замену цепи с сальниками (O-Ring) и звёзд я выполнил после 26 тысяч
пробега;
\item передачи переключаются нечётко, а словить нейтраль порой
становится непосильной задачей. Особенно радует особенность некоторых
мотоциклов, когда завести его можно только на нейтрали. Это просто
песня какая-то --- заглохнуть посреди перекрёстка и судорожно
искать непослушную нейтраль только ради того, чтоб завести двигатель;
\item всё, что может быть сделано из пластмассы, делается из пластмассы
и даже рассеиватели фар - не исключение. Домашнее задание: при покупке
нового китайского мотоцикла, проверьте, из какого материала изготовлены
тормозные колодки;
\item хрупкий пластик. Даже то, что, по идее, должно при деформациях
максимум гнуться, трескается и раскалывается;
\item если на байке есть ``хромированный'' декор, то с большой
вероятностью он пластиковый, который со временем начинает затираться
или с него начинает слезать ``хром''. Если декор таки металлический,
спустя считанные недели он начинает совершенно наглым образом
покрываться ржавчиной;
\item кофры, идущие в комплекте прямо от дилера, запросто могут по
своей геометрии не годиться для установки, например, не висеть, а
лежать на указателях поворотов;
\item совершенно не светит обзавестись руководством по ремонту и
обслуживанию --- иногда в Internet даже невозможно найти
релевантное упоминание про модель мотоцикла, не говоря уже про сайт
производителя. Эта беда --- полубеда. Ремонт в случае чего, конечно,
ему дать можно, поскольку никаких космических технологий там не
применяется и при наличии определённых навыков и умений можно
исправлять проблемы и без мануала. Если навыков и умений нет, то
надо искать мастера и здесь ждёт ещё один сюрприз --- многие СТО,
специализирующиеся на обслуживании мотоциклов, прямо заявляют,
что они \emph{не занимаются китайцами};
\item ещё одна проблема, связанная с невозможностью определить
производителя --- это поиск запчастей. У всяких хонд и иже с ними
каждый винтик для каждой модели каждого года имеет свой уникальный
номер, который можно найти по справочникам в Internet и заказать
его, будучи уверенным, что тебе придёт именно то, что надо. Для
Noname`ов же поиск запчастей необходимо выполнять вручную.
\item видимо, у каждого производителя China-байков в штате есть ``инженер
по стуку задней полки'', благодаря которому при езде хоть что-нибудь,
но обязательно будет дребезжать;
\item и напоследок: мотоцикл --- это вещь не на всю жизнь и рано или
поздно станет вопрос о его продаже. По моему опыту, желающих купить
китайца значительно меньше желающих купить японца/немца и т.п. К
примеру, после размещения объявления о продаже японского спортбайка,
которому перевалило за 13 лет, звонки пошли сразу же и на третий день
его забрали. А вот после размещения объявления по продаже практически
нового китайца (только успел пройти обкатку) не поступило
\emph{ни одного звонка}.
\end{itemize}

К слову о китайцах. В последние годы они откровенно обнаглели и
штампуют клоны известных японских мотоциклов, меняя в названии одну-две
буквы. С расстояния в пару метров оно смотрится, как японец, но стоит
только подойти и посмотреть вооружённым глазом, правда бросается в
глаза.

Если повезёт с клоном, то его потихоньку, по одной-две детали можно
полностью заменить на японца, поскольку все детали --- один в один с
оригиналом. Такой случай описан одним владельцем China-байка ---
клона Yamaha Virago. Правда, обычно клоны с оригиналом не имеют ничего
общего, кроме внешнего сходства.

\subsubsection{Покупка с рук}

В принципе, ничего страшного или постыдного, только надо помнить,
что кругом обман и кидалы на вторичном рынке --- вещь достаточно
распространённая, к сожалению. Дело лучше иметь с людьми, о которых
ты хоть что-то знаешь. Общие правила такие:

\begin{itemize}
\item грамотный осмотр и оценка техсостояния;
\item никаких предоплат или авансов (частая схема кидков);
\item на слово никому не верить.
\end{itemize}

Торг уместен всегда. Если ты сможешь при осмотре распознать косяк,
про который тактично умолчал хозяин, то это повод скостить цену на
стоимость исправления косяка. Царапины тут не в счёт (есть даже такое
правило: непадавших мотоциклов не бывает), а вот такие вещи как сопливая
или погнутая вилка, стёртые звёзды и растянутая цепь (-250\$),
потрескавшаяся резина (-300\$) --- всё это влетит в копеечку уже новому
хозяину.

\subsection{Смена владельца или доверенность}

Некоторые, экономя на процедуре снятия/постановки транспортного
средства на учёт, прибегают к такой махинации, как покупка по
доверенности: продавец оформляет у нотариуса генеральную доверенность
на покупателя, покупатель отдаёт продавцу деньги и все расходятся.

Подводные грабли этого метода:

\begin{itemize}
\item законным владельцем транспортного средства остаётся прежний
хозяин;
\item законный владелец транспортного средства в любой момент может
аннулировать такую генеральную доверенность (и в качестве бонуса
объявить мотоцикл в угон);
\item в случае смерти законного владельца доверенность автоматически
теряет силу;
\item согласно законодательству Украины в некоторых случаях
ответственность за причинённый ущерб распространяется не только на того,
кто был за рулём, но и на законного владельца транспортного средства.
\end{itemize}

\subsection{С документами или без?}

Здесь речь пойдёт о так называемых ``бездоках'' --- мотоциклах с мутным
прошлым, настоящим и будущим, не поставленным на учёт в МРЕО, без
документов, удостоверяющих право собственности и т.п. Бездок на
вторичном рынке будет стоить 30\%-40\% от стоимости аналогичного,
но легального мота. Покупая бездок, важно понимать:

\begin{itemize}
\item сделка купли-продажи никем и нигде не фиксируется;
\item нет гарантии того, что мот не был угнан час назад за углом
соседнего дома + неизвестно, у кого именно он был угнан;
\item гайцы тоже будут очень рады. По дорогам они за тобой гоняться
не будут, они ловят таких умельцев на заправках и стоянках.
\end{itemize}

И небольшая невыдуманная история в заключение. Один мой товарищ А. купил
бездоковый мот у своего плохого знакомого С. Через некоторое время,
не успев даже на нём покататься, решил его продать. Разместил
объявление в Internet, встретились с покупателем, обменялись ключами
и деньгами. Проходит неделя и выясняется, что новый хозяин тоже
решил выставить его на продажу, посмотреть пришло несколько
типочков, которые вставили свой ключ в мотик и завели его. Набили
лицо продавцу, мол, год назад у нас его угнали и уехали в неизвестном
направлении. Разборки, разумеется, докатились и до моего товарища А.,
который, формально, хозяином уже неделю как не являлся.

Итог: не стоит связываться с бездоком, поскольку даже если брать его
как донора для разборки в гараже, то не факт, что за ним не придут с той
стороны, где он был куплен (милиция, к примеру).

\subsection{К мотоциклу}

Если будет храниться на стоянке под открытым небом, нужен чехол. За
200 грн можно взять промокаемую тряпку, которая к тому же может ещё и
загореться/поплавиться о горячую выхлопную трубу. Нормальный чехол
можно взять за 500 грн и выше. Ну и про соответствие размеров мотика
и чехла не забываем.

Противоугонные средства. Самые простые --- это цепь (для пристёгивания
к капитальным конструкциям) и блокировка колёс. Тут надо понимать, как
мотики угоняют. Если мотик не тяжёлый и не пристёгнут к чему-то
капитальному за раму, то за считанные секунды подъезжает бус, два-три
поца его хватают, запихивают внутрь и всё --- нет у тебя больше мота.
Заблокированное колесо, как ты понимаешь, от таких негодяев не спасёт.
Также можно поставить и сигнализацию --- начнёт орать, как только его
кто-то попытается выровнять или покатить. Призвание сигнализации ---
привлечь внимание. Самому факту кражи она, естественно, препятствует
лишь косвенно.

\section{В дорогу!}

\subsection{Особенности езды по городу}

Правила дорожного движения для всех едины, но всегда надо помнить:
что автомобилисту помятость, тебе --- падение и увечье или смерть.
Для мотоциклиста на дороге \emph{все} являются врагами.
Держись своей полосы. Всегда обозначай манёвры поворотниками.
Езди так, чтоб тебя видело как можно больше участников дорожного
движения, а именно --- ближе к краю полосы. Плюсы: ты повторяешь
траекторию движения колёс впереди идущего автомобиля и не едешь
посередине полосы и потому у тебя меньше шансов наехать на прятствие,
которое впереди идущий автомобиль пропустит между колёсами; тебя
видят водители не только впереди и сзади идущего, но многие из
соседней полосы; даже в случае экстренного торможения у тебя
появляется шанс уйти в междурядье, вместо того чтобы ляпнуться
кому-то в зад; в конце-концов, ты и сам гораздо лучше видишь
происходящее на дороге.
Труднее всего соблюдать безопасную дистанцию, потому что постоянно
кто-то норовит заполнить её собой.

Всегда будь готов к тому, что в тебя начнут перестраиваться или
что кто-то будет выезжать на главную, не предоставив тебе
преимущество. Причина проста --- тебя не видят.

Не забываем и про то, что фуры имеют право
выполнять повороты не из крайнего положения. Один раз я так еле
разминулся с фурой, которая пошла на разворот.

Как показывает моя практика, езда по первой полосе --- это далеко
не всегда наилучший выбор. Во-первых, она очень часто убита маршрутками,
троллейбусами и т.п. Во-вторых, не редкий случай, когда тело со второй
полосы без обозначения манёвра перестраивается в твою первую полосу
перед тобой и останавливается. А встречаются и товарищи, выполняющие
поворот не из крайнего положения без включения поворотников.

Пару слов про езду по междурядью. Не будем скрывать, это очень не
нравится водителям автомобилей. Они начинают нервничать, переживая
за царапины на боках и побитые зеркала. Лично я еду по междурядью со
скоростью не более 20 км/ч, т.е. только в тех случаях, когда поток
практически стоит. По словам участников, даже при скорости 30 км/ч
уже запросто можно не успеть вовремя затормозить, когда автомобиль
решит перестроиться или кто-то решит открыть дверь. При таком
шнырянии рекомендуется не катиться тихонько на нейтрали, а ехать
на пониженной, чтоб водители знали о твоём приближении по рёву
двигателя.

\subsection{Особенности езды за городом}

Лучше держаться правой стороны, если это позволяет покрытие.
С левой стороны очень любят носиться со скоростью 130 и более км/ч,
при этом порой они не дожидаются, пока ты сместишься вправо, а
проносятся в сантиметрах от тебя. Сказать, что это неприятное ощущение ---
это ничего не сказать, особенно если ты заметил его только после факта
опережения. Скоростной режим я бы посоветовал 90+, не взирая на то,
что по ПДД ограничение скорости для мотоциклов даже на магистралях
80 км/ч. Обоснование --- с такой скоростью идут фуры. Лучше уж их обгонять,
чем они будут обгонять тебя. Да и вообще, чем меньше твоя скорость,
тем больше тебе надо следить по зеркалам заднего вида. Это утомляет.
Ну и не забываем о том, что на дорогах общего пользования 150 для
мотоцикла --- это уже совсем не безопасная скорость и в случае
падения велика вероятность склеить ласты (всё, конечно, зависит от
обстоятельств и, как показывает статистика, иногда и 30 хватает).

\subsection{Подставы на дорогах}

Об этом, к сожалению, книг не пишут. Однажды я в плотном потоке
машин попал в очень неприятную ситуацию, из которой выбрался только
чудом. ``Совпадение'', --- подумал я и похвалил себя за то, что вышел
сухим из воды. Спустя некоторое время я чисто случайно смотрел
ролик на YouTube про подставы на дорогах и увидел своё совпадение
один в один. Стало не по себе --- одно дело, когда это просто
случайность, а другое дело, когда твоей шкурой начинают играть
какие-то уроды.

Совет: смотреть тематические ролики с видео-регистраторов до
просветления.

\subsection{Разводняки}

Ещё одна тема, о которой не пишут.

Еду по трассе, стоит на обочине машина, водила голосует.
Я остановился, спрашиваю, чо надо. Человек на ломаном русском
начинает просить денег на бенз, бо остался вдали от дома, предлагая
взамен свои кольца-перстни. Я железо брать категорически отказывался,
но сказал, что могу помочь на 200 грн. Вручил ему, уже тронулся с
места, вижу, он догоняет. Добежал, вручил кольцо со словами ``сувенир''.

Приехал я в населённый пункт, захожу в ломбард, спрашиваю: ``Что это?''.
Человек за стойкой сходу заявляет: ``Латунь. Её чурки на дороге парят''.

Вот это и называется --- смешанные чувства. С одной стороны, мудила
воспользовался моей добродушностью и сделал меня лохом. И теперь хрен
я помогу человеку, даже если ему, возможно, нужна будет помощь. С другой
стороны, мудила таки напарил мне свой перстень, чтобы я не тешил себя
иллюзиями, а таки узнал, что я лох :)

Итог отсюда: надо очень внимательно относиться к людям, просящим на
большой дороге. А в безлюдных местах я бы вообще не останавливался.

\clearpage

\section{Что дальше?}

Ниже приведен список хорошо зарекомендовавшей себя литературы на
мото-тематику и несколько полезных ссылок, с помощью которых можно
успешно продолжить ознакомление с предметной областью.

\subsection{Особо рекомендуемая литература}

\begin{itemize}
\item Дэвид Л. Хафф, Искусство вождения мотоцикла
\item Дэвид Л. Хафф, Дорожная стратегия
\item Ли Паркс, Полный контроль
\end{itemize}

\subsection{Ссылки}

\begin{itemize}
\item \url{http://moto.kiev.ua/} --- флагманский форум на мототематику в стране;
\item \url{http://www.motosale.com.ua/} --- весьма раскрученный сайт по покупке/продаже мототехники;
\item \url{http://www.motodom.ua/books.html} --- сборник мото-литературы.
\end{itemize}

\clearpage

\section{Благодарности}

Благодарю за содействие (явное либо косвенное) в написании статьи:

\begin{itemize}
\item Рома на клоне Yamaha Virago 250cc - за приятную беседу на дороге
в ожидании окончания ливня;
\item Олег Гордиенко --- за предоставленные в пользование мотоциклы;
\item Виталий Кириченко --- за весёлые и познавательные истории;
\item Вадим Хоменко --- за ответы на все мои заданные вопросы;
\item Михаил Харченко --- за дельные советы;
\item Игорь Клименко, Данило Павляк, Михаил Тыж --- за идею оформить
мои эмпирические наблюдения в виде статьи.
\end{itemize}

\end{document}
